\documentclass[10pt]{article}
\usepackage[fleqn]{amsmath}
\usepackage{titlesec}
\usepackage{multicol}
\usepackage{ifthen}
\usepackage[landscape]{geometry}
\usepackage{nicefrac}
\usepackage{etoolbox}

% To make this come out properly in landscape mode, do one of the following
% 1.
%  pdflatex latexsheet.tex
%
% 2.
%  latex latexsheet.tex
%  dvips -P pdf  -t landscape latexsheet.dvi
%  ps2pdf latexsheet.ps

% This sets page margins to .5 inch if using letter paper, and to 1cm
% if using A4 paper. (This probably isn't strictly necessary.)
% If using another size paper, use default 1cm margins.
\ifthenelse{\lengthtest { \paperwidth = 11in}}
    { \geometry{top=.5in,left=.5in,right=.5in,bottom=.5in} }
    {\ifthenelse{ \lengthtest{ \paperwidth = 297mm}}
    {\geometry{top=1cm,left=1cm,right=1cm,bottom=1cm} }
    {\geometry{top=1cm,left=1cm,right=1cm,bottom=1cm} }
    }

% Turn off header and footer
\pagestyle{empty}
 
% Put footnotes at end of document
% \let\footnote=\endnote

% Redefine section commands to use less space
\titleformat{\section}{\centering\large\bfseries}{}{0em}{}
\titlespacing{\section}{0pt}{0.5ex plus 0.2ex}{-2ex plus -0.5ex minus%
-0.2ex}
\titleformat{\subsection}{\centering\normalsize\bfseries}{}{0em}{}
\titlespacing{\subsection}{0pt}{0.5ex plus 0.2ex}{0ex plus -0.5ex%
minus 0.2ex}
\titleformat{\subsubsection}{\centering\small\bfseries}{}{0em}{}
\titlespacing{\subsubsection}{0pt}{0.5ex plus 0.2ex}{-2ex plus -0.5ex%
minus 0.2ex}

% Don't print section numbers
\setcounter{secnumdepth}{0}

\setlength{\parindent}{0pt}
\setlength{\parskip}{0pt plus 0.5ex}
\setlength{\mathindent}{0pt}

\newcommand{\scinot}[2]{\ensuremath{#1 \times 10^{#2}}}
\newcommand{\id}[1]{\mathrm{d} #1}
\newcommand{\bvec}[1]{\vec{\mathbf{#1}}}
\newcommand{\buvec}[1]{\hat{\mathbf{#1}}}
\DeclareMathOperator{\sign}{sign}

% -----------------------------------------------------------------------

\begin{document}

\raggedleft
\footnotesize
\begin{multicols*}{5}

% multicol parameters
% These lengths are set only within the two main columns
%\setlength{\columnseprule}{0.25pt}
\setlength{\premulticols}{1pt}
\setlength{\postmulticols}{1pt}
\setlength{\multicolsep}{1pt}
\setlength{\columnsep}{2pt}

\begin{center}
    \Large
    Physics \textrm{II} \\
    Formula Sheet
\end{center}

\section{Constants}

\begin{alignat*}{2}
    c   &= \scinot{2.998}{8}   && \quad \mathrm{m / s}\\
    q_e &= \scinot{1.602}{-19} && \quad \mathrm{C}\\
    m_e &= \scinot{9.109}{-31} && \quad \mathrm{kg}\\
    \mu_e &= \scinot{-9.285}{-24} && \quad \mathrm{J/T}\\
    k   &= \scinot{8.988}{9}   && \quad \mathrm{N \cdot m^2 \cdot
    C^{-2}}\\
    \varepsilon_0 &= \frac{1}{4\pi k} \\
        &= \scinot{8.854}{-12} && \quad \mathrm{F \cdot m^{-1}}\\
	\mu_0 &= \scinot{1.257}{-6} && \quad \mathrm{T \cdot m/A}\\
    A_v &= \scinot{6.022}{23} && \quad (\text{None})
\end{alignat*}

\section{Electricity}

\begin{align*}
    \vec{E}_{iP} &= \frac{kq_i}{r_{iP}^2} \hat{\mathbf{r}}_{iP}\\
    V &= \frac{kq}{r}\\
    U &= qV = \frac{kq'q}{r}\\
    U &= \frac{1}{2} \sum_{i=1}^n q_i V_i\\
    U &= \frac{1}{2} QV
\end{align*}

\subsection{Electric Dipoles}
\begin{align*}
    \vec{p} &= q\vec{L}\\
    \vec{E} &= \frac{2kp}{\|x\|^3} \quad (x \to \infty)\\
    \vec{\boldsymbol{\tau}} &= \bvec{p} \times \bvec{L}\\
    U &= -pE\cos \theta + U_0\\
    U &= -\bvec{p} \cdot \bvec{E} \quad (\text{if $U_0 = 0$})\\
    V &= \frac{kp}{x^2} \quad (x \gg \ell)
\end{align*}

\section{Magnetism}

\begin{align*}
    \vec{F} &= q\vec{v} \times \vec{B} && (\text{Lorentz Force})\\
      &= I\vec{L} \times \vec{B}\\
    T &= \frac{2\pi m}{qB} && (\text{Period})\\
    \omega &= 2\pi f = \frac{q}{m}B\\
    \mathcal{E} &= -\frac{\id{\phi_m}}{\id{t}} && (\text{Faraday's
	Law})\\
                &= -L \frac{\id{I}}{\id{t}}\\
    \vec{B} &= \frac{\mu_0}{4\pi} \frac{q\vec{v} \times \buvec{r}}
    {r^2}\\
    \id{\vec{B}} &= \frac{\mu_0}{4\pi} \frac{I\id{\vec{\ell}} \times
    \buvec{r}} {r^2} && (\text{Biot-Savart Law})\\
    V_\mathrm{H} &= E_\mathrm{H} w \\
    &= \upsilon_\mathrm{d}Bw && (\text{Hall Voltage})\\
    n &= \frac{|I|B}{tq_eV_\mathrm{H}}
\end{align*}

\subsection{Magnetization}

\begin{align*}
    \vec{M} &= \frac{\id{\vec{\mu}}}{\id V} =
    \frac{\id{\vec{I}}}{\id{\ell}}\\
    \vec{B} &= \vec{B}_\text{app} + \mu_0 \vec{M}\\
	    &= \vec{B}_\text{app} (1 + \chi_m)\\
	    &= \vec{B}_\text{app} K_m
\end{align*}

\subsection{Magnetic Dipoles}

\begin{align*}
    \vec{\mu} &= NIA \buvec{n}\\
    \vec{\tau} &= \vec{\mu} \times \vec{B}\\
    U &= -\mu B \cos \theta + U_0\\
      &= -\vec{\mu} \cdot \vec{B}
\end{align*}

\subsection{The Mass Spectrometer (Cyclotron)}

\begin{align*}
    v &= \frac{rqB}{m}\\
    \frac{m}{q} &= \frac{(Br)^2}{2 |\Delta V|}\\
    K &= \frac{1}{2} m v^2 = \frac{1}{2}
    \frac{\left({qBr}\right)^2}{m}
\end{align*}

\subsection{Coils}

\begin{align*}
    B &= \frac{1}{2} \mu_0 nI \left({\frac{z-z_1}{\sqrt{(z-z_1)^2 +
	    R^2}} - \frac{z-z_2}{\sqrt{(z-z_2)^2 + R^2}}}\right)\\
      &= \frac{1}{2} \mu_0 nI \frac{\ell}{\sqrt{(\ell/2)^2 + R^2}}
	  \quad (\text{Solenoid})\\
      &\approx N\mu_0 I\\
    L &= \mu_0 n^2 A \ell\\
    B &= \frac{\mu_0NI}{2\pi r} \qquad (\text{Toroid})\\
    \phi_m &= NBA \cos \theta
    % \footnote{$\theta$ is the angle between the normal vector and the
	% field.}
\end{align*}

\section{Circuits}

\begin{align*}
    \hspace{2em} V &= IR && (\text{Ohm's Law})\\
    P &= IV = I \mathcal{E}\\
    V_T &= \mathcal{E} - Ir\\
    E &= Q \mathcal{E}\\
    \bvec{J} &= qn\vec{\upsilon}_\mathrm{d}\\
    I &= \pm JA\\
    R &= \rho \frac{L}{A}\\
    \frac{1}{C_{\text{eq}}} &= \frac{1}{C_1} + \cdots &&
    (\text{Series})\\
    C_{\text{eq}} &= C_1 + \cdots && (\text{Parrallel})\\
    \tau &= RC && (\text{RC Circuit})\\
    \tau &= L/R && (\text{RL Circuit})
\end{align*}

\section{Line}

\begin{align*}
    E_x &= \frac{kQ}{x^2} \quad (x \gg L)\\
    B &= \frac{\mu_0}{4\pi} \frac{I}{R} \left({\sin \theta_2 - \sin
	    \theta_1}\right)
\end{align*}

\subsection{Infinite Line}

\begin{align*}
    E &= \frac{2k\lambda}{R}\\
    V &= 2k\lambda \ln \frac{R_\text{ref}}{R}\\
    B &= \frac{\mu_0 I}{2\pi R}\\
    \frac{\id{F_{12}}}{\id{\ell}} &= \frac{\mu_0}{2\pi} \frac{I_1 I_2}
    {R} && (\text{The Amp\`{e}re})
\end{align*}

\vspace*{20ex}
\section{Ring}

\begin{align*}
    E &= \frac{kQx}{\left({x^2 + a^2}\right)^{3/2}}\\
    V &= \frac{kQ}{\sqrt{x^2 + a^2}}\\
    \mu &= I\pi R^2\\
    B &= \frac{\mu_0IR^2}{2\left({x^2 + R^2}\right)^{3/2}}\\
	  &= \frac{\mu_0 I}{2R} \quad (x = 0)\\
      &= \frac{\mu_0 \mu}{2\pi |x|^3} \quad (x \gg R)\\
    B &= 
    \begin{cases}
	\frac{\mu_0I}{2\pi R^2} r & \quad r \leq R\\
	\frac{\mu_0}{2\pi} \frac{I}{r} & \quad r \geq R
    \end{cases}\\
\end{align*}

\section{Disk}

\begin{align*}
    E &= \sign x \cdot 2\pi k \sigma \left({1 - \frac{1}{\sqrt{1 +
    (R/x)^2}}}\right)\\
      &= \sign x \cdot 2\pi k \sigma \quad (R \to \infty)\\
    V &= 2\pi k \sigma |x| \left({\sqrt{1 + (R / x)^2} - 1}\right)
\end{align*}

\section{Infinite Plane}

\begin{align*}
    E_x &= \sign x \cdot \frac{\sigma}{2\varepsilon_0}\\
    V_x &= V_0 - 2\pi k \sigma |x|
\end{align*}

\section{Spherical Shell}

\begin{align*}
    E_r &=
    \begin{cases}
    k \frac{Q}{r^2} & \quad r > R\\
    0 & \quad r < R
    \end{cases}\\
    V &=
    \begin{cases}
    \frac{kQ}{r} & \quad r \geq R\\
    \frac{kQ}{R} & \quad r \leq R
    \end{cases}
\end{align*}

\section{Solid Sphere}

\begin{align*}
    E_r &=
    \begin{cases}
    k \frac{Q}{r^2} & \quad r \geq R\\
    k \frac{Q}{R^3} r & \quad r \leq R
    \end{cases}\\
    V &=
    \begin{cases}
    \frac{kQ}{r} & \quad r \geq R\\
    \frac{kQ}{2R} \left({3 - \left({\frac{r}{R}}\right)^2}\right) &
    \quad r \leq R
    \end{cases}
\end{align*}

\section{Capacitance}

\begin{align*}
    C &= Q / V\\
    C &= \frac{\varepsilon_0 A}{d} && (\text{Parrallel Plate})\\
    C &= \frac{2\pi \varepsilon_0 L}{\ln (R_1 / R_2)} &&
    (\text{Cylindrical})\\
    \varepsilon &= \kappa \varepsilon_0\\
    u_e &= \frac{1}{2}\varepsilon_0 E^2
\end{align*}

\section{Inductance}

\begin{align*}
    \phi_m &= LI\\
    U_m &= \frac{1}{2} LI^2\\
    u_m &= \frac{B^2}{2\mu_0}\\
    L &= \mu_0 n^2 A \ell && (\text{Solenoid})
\end{align*}

\section{EM Waves}

\begin{align*}
    E &= cB\\
    u &= u_e + u_m = 2u_e\\
      &= \nicefrac{EB}{\mu_0 c}\\
    I &= u_\text{av}c\\
      &= \frac{E_\text{rms}B_\text{rms}}{\mu_0}\\
    \vec{S} &= \frac{\vec{E} \times \vec{B}}{\mu_0} && (\text{Poynting
	Vector})\\
    P_r &= \nicefrac{I}{c} && (\text{Radiation Pressure})
\end{align*}

\section{Maxwell's Equations}

\setbool{@fleqn}{false}
\begin{gather*}
    \nabla \cdot \vec{E} = \frac{\rho}{\varepsilon_0}\\
    \nabla \cdot \vec{B} = 0\\
    \nabla \times \vec{E} = -\frac{\partial \vec{B}}{\partial t}\\
    \nabla \times \vec{B} = \mu_0 \left({\vec{J} + \varepsilon_0
	    \frac{\partial \vec{E}}{\partial t}}\right)
\end{gather*}

\end{multicols*}
\newpage
% --------------------------------------------------------------------
\begin{multicols*}{4}
\begin{center}
    \Large
    Physics \textrm{I} \\
    Formula Sheet
\end{center}

\subsection{Free Fall}

\begin{align*}
    h &= \frac{1}{2}gt^2\\
    v &= gt\\
      &= \sqrt{2hg}
\end{align*}

\section{Force}

\begin{align*}
    F_f &= \mu F_N && (\text{Friction})\\
    F &= m \frac{v^2}{r} && (\text{Centripital})
\end{align*}

\section{Rotation}

\begin{align*}
    \omega     &= \omega_0 + \alpha t\\
    \theta     &= \frac{\omega_i + \omega_f}{2} t\\
    \theta     &= \omega_it + \tfrac{1}{2}\alpha t^2\\
    \omega_f^2 &= \omega_i^2 + 2 \alpha \theta\\
    v          &= r\omega\\
    a          &= r\alpha\\
    T          &= \frac{2\pi r}{v}\\
    \omega     &= 2\pi f\\
    \tau       &= I\alpha\\
    L          &= I\omega\\
    W          &= \tau \theta\\
    K          &= \tfrac{1}{2}I\omega^2
\end{align*}

\section{Pressure}

\begin{align*}
    \rho &= \frac{m}{V}\\
    P &= \frac{F}{A}\\
    \Delta P &= \rho gh\\
    \frac{F_1}{A_1} &= \frac{F_2}{A_2} && (\text{Pascal's
	Principle})\\
    Q &= A\nu && (\text{Flow Rate})\\
    Q_1 &= Q_2\\
    F_B &= \rho Vg\\
    P_1 &+ \frac{1}{2}\rho \nu_1^2 + \rho gy_1\\
	&= P_2 + \frac{1}{2}\rho \nu_2^2 + \rho gy_2 &&
	(\text{Bernoulli's Equation})
\end{align*}

\section{Sound}

\begin{align*}
    I &= \frac{P}{A}\\
      &= \frac{P}{4\pi r^2}\\
    \beta &= 10 \log \left({\frac{I}{I_o}}\right)
\end{align*}

\section{Simple Harmonic Motion}
\vspace{2ex}
\subsection{General}

\begin{align*}
    x &= A \cos (\omega t + \phi)\\
    k &= m\omega^2\\
    U &= \tfrac{1}{2} kx^2\\
    E &= \tfrac{1}{2} kA^2\\
    T &= 2\pi \sqrt{\frac{m}{k}}\\
    v_x &= \pm \sqrt{\tfrac{k}{m} (A^2 + x^2)}
\end{align*}

\subsection{Pendulum}

\begin{align*}
    T &= 2\pi \sqrt{\frac{l}{g}}\\
    T &= 2\pi \sqrt{\frac{I}{mgd}}
\end{align*}

\subsection{Waves}

\begin{align*}
    v &= f\lambda\\
    y &= A \sin 2\pi \left({\frac{t}{T} - \frac{x}{\lambda}}\right) &&
    +x\\
    y &= A \sin 2\pi \left({\frac{t}{T} + \frac{x}{\lambda}}\right) &&
    -x
\end{align*}

\end{multicols*}
\end{document}

